\documentclass[a4paper,11pt]{article}
\usepackage[utf8]{inputenc}
\begin{document}
\noindent
\begin{center}
\Large\textbf{RITS Principles}
\end{center}

\vspace{1cm}

\begin{enumerate}
\item \textbf{RITS is here to help.}\\
  The most important goal of RITS is to help students, teachers,
  administrators, parents, and all people involved in any way with
  education.

  \bigskip\noindent
  RITS helps \textit{students} to\dots
  \begin{itemize}
  \item \textit{understand} the subject material
  \item \textit{assess} their own knowledge and skills
  \item \textit{detect} and correct their misconceptions and blind spots.
  \end{itemize}

  \bigskip\noindent
  RITS helps \textit{teachers} to\dots
  \begin{itemize}
  \item \textit{reduce} their workload related to correcting and grading
    assignments, reporting individual and aggregated progress, \dots
  \item \textit{focus} on the many important tasks that will always require a
    human person: designing the curriculum and selecting assignments,
    connecting knowledge from different areas, explaining current
    events and pressing issues in today's world
  \end{itemize}
  \bigskip\noindent
  
  RITS helps \textit{administrators} to\dots
  \begin{itemize}
  \item \dots
  \end{itemize}

\item \textbf{The way is the goal, and vice versa.}\\
  Grades and assessments are only a means to an end, but not an end by
  themselves.

  Every interaction with RITS should lead to a better
  \textit{understanding} of the subject matter. RITS always
  \textit{accepts} a correct answer. When an answer is correct but not
  yet optimal, such as $\frac{2}{4}$ in the case of canceling
  fractions, RITS accepts the answer \textit{and leads the student} to
  the optimal solution.


\item \textbf{Balance is key.}\\ % find a more fitting slogan for this?
  RITS is here to help, not frustrate. Guidance is given when
  necessary, by interjecting and solving subproblems. These
  subproblems can be skipped by students, because erroneous answers
  may be due to simple mistypes that need no lengthy interludes to
  correct. Students should be able to explore the system as freely as
  possible, if they desire to do so, while care is taken not to lose
  them to trial-and-error methods.
  \end{enumerate}

\vfil
\end{document}

