\documentclass[a4paper,11pt]{article}
\usepackage[utf8]{inputenc}
\usepackage{graphicx}
\newcommand{\pslisting}[1]{\hrule\vskip0.2cm\includegraphics{#1}\vskip0.2cm\hrule}
\begin{document}
\noindent
\begin{center}
\Large\textbf{RITS System Architecture}
\end{center}

\vspace{1cm}
\noindent
The Rule-based Intelligent Tutoring System (RITS) contains and
controls all logic for interacting with users. Each client (which we
use as a synonym for ``user'' throughout) sends user actions to RITS,
and then queries RITS for the next action to be performed. To use
RITS, add the following line to your Prolog code:

\bigskip
\textbf{:- use\_module(rits).}
\bigskip

\noindent RITS is accessible via the following API calls:

\begin{itemize}
\item \textbf{rits\_start(-Start)}\\\textbf{Start} is unified with a
  Prolog term that represents the initial state of RITS, before the
  system has received any client (i.e., user) actions.

\item \textbf{rits\_next\_action(+A0, -A, +S0, -S)}\\
  This predicate relates a client action \textbf{A0} and a RITS state
  \textbf{S0} to the next RITS action \textbf{A} and next
  state~\textbf{S}.

\item \textbf{rits\_history(+S, -Hist)}\\
  \textbf{Hist} is unified with a list of Prolog terms that represent
  the history of client interactions up to and including state~\textbf{S}.
\end{itemize}

These predicates are completely \textit{pure}: They do not emit any
output, do not write any files, and do not query the user in any way.
Therefore, they can be quite safely executed in a hosting environment.

Client actions are described in Table~\ref{tab:useractions}. The most
important RITS actions are described in Table~\ref{tab:ritsactions}.
RITS clients are expected to interpret RITS actions ``appropriately'',
such as: displaying messages on the terminal or via HTML, allowing
users to give answers etc.

\begin{table}[ht]
  \centering
  \begin{tabular}{cp{8cm}}
    \hline
    \textbf{solve(Task)} & Guide the student through solving \textbf{Task}.\\
    \textbf{student\_answers(T)} & Tell RITS that the student has responded with the Prolog term~\textbf{T}. This action is only admissible if the directly preceding RITS action was \textbf{read\_answer}.\\
    \textbf{next} & Query RITS for its next action. \\
    \textbf{skip} & The student wants to skip solving the subproblem that is
    currently in progress. (\textit{TODO})\\
    \hline
  \end{tabular}
\caption{Admissible client actions}
  \label{tab:useractions}
\end{table}

\begin{table}[ht]
  \centering
  \begin{tabular}{cp{8cm}}
    \hline
    \textbf{enter} & Start of a subproblem that the student must solve.\\
    \textbf{exit} & The most recently spawned subproblem was solved.\\
    \textbf{format(F)} & The string \textbf{F} is to be displayed.\\
    \textbf{format(F,As)} & The format string \textbf{F} is to be displayed. \textbf{As} is a list of arguments that are output, in sequence, in place of each \texttt{\~\ $\!\!\!$w} that appears in \textbf{F}.\\
    \textbf{read\_answer} & The student is to be queried for a an answer.\\
    \textbf{solve(Task)} & The student should be given the task \textbf{Task}.\\
    \textbf{done} & There are no further actions.\\
    \hline
  \end{tabular}
\caption{The most important RITS actions}
  \label{tab:ritsactions}
\end{table}


\vspace{0.7cm}

Due to their purity, these predicates are also very amenable to
regression testing and further analysis. (\textit{more to follow})

\vspace{0.7cm}

The RITS engine may be accessed directly via Prolog, remotely via
JavaScript and \textit{pengines}, or via any other language embedding.

\bigskip
\begin{center}
  \large\textbf{Extending RITS}
\end{center}

\medskip
\noindent Internally, RITS uses the following predicates to decide
what to do. All of them are DCG rules that describe a list of RITS
actions that need to be performed in response to certain client actions.

\begin{itemize}
\item \textbf{rits:solve//1}: this is called with argument \textbf{Task}
  for client actions of the form \textbf{solve(Task)}.

\item \textbf{rits:actions//3}: This is called with arguments
  \textbf{Task} and \textbf{Answer} after the user (i.e., client)
  answered the question posed after \textbf{solve(Task)} with
  \textbf{Answer}. The third argument is a Prolog representation of
  previous interactions of the form \textbf{T=A}, in reverse
  chronological order.

\end{itemize}

RITS can be extended in a \textit{modular} way to handle new domains.
To teach RITS additional rules, define a Prolog module that provides
its own (additional) DCG rules for \textbf{rits:solve//1} and
\textbf{rits:actions//3}. 

In these DCG rules, modules may provide their own custom RITS actions
as Prolog terms that must be handled by the client when they appear as
a RITS action upon calling \textbf{rits\_next\_action/4}.

See the file \textbf{rits\_multiple\_choice.pl} for the definition of
simple multiple choice tests that are repeated when wrong answers are
given.

\pagebreak

\begin{center}
  \large\textbf{LORITS: The extensible language of RITS}
\end{center}

\medskip
\noindent
Preparing teaching material and exercises for students is one of the
most common and essential tasks that teachers need to perform
every day. It requires judicious human creativity and knowledge and
cannot be automated.

Describing the material that is to be taught is a \textit{declarative}
task: We primarily want to describe \textit{what} should be presented
to students. \textit{How} it is presented can vary by circumstances.
For example, on a text terminal, we expect a presentation of the
material that differs from that on mobile phones, Braille terminals,
and sheets of paper.

RITS makes it easy for you to declaratively describe \textit{what}
should be done. RITS comes with built-in modules that describe what it
means for students to ``solve'' a task like canceling fractions. In
addition, you can supply your own rules for RITS, thereby stating what
you mean by custom tasks that you define for your students.

The language of RITS is called LORITS and lets you state actions that
need to be performed in order for students to ``solve'' a task. For
example, it is easy to declaratively describe your ``Lesson 1'', which
may consist of showing a video that contains some facts, and then have
students automatically go through a custom multiple-choice test that
asks them questions about the video in order to assess their
understanding:

\begin{verbatim}
 rits:solve(lesson_1) -->
         [video("http://www.youtube.com/VideoAboutMozambique"),
          solve(mchoice("What is the capitol of Mozambique?\n",
                        ['Maputo','Pretoria',
                         'Nairobi','Vienna'], 1)),
          done].
\end{verbatim}

LORITS provides predefined actions such as formatting texts, going
through multiple-choice tests, playing videos etc., which makes it
easy to put together custom lessons that students can work through.

\bigskip \textit{Here, it is our task to provide teachers with
  suitable building blocks of RITS actions, so that custom lessons can
  be put together quickly, possibly with our initial assistance. The
  task here is to find good and versatile actions to cover a large
  variety of interaction styles.}

\pagebreak

\begin{center}
  \large\textbf{UTRITS: Expressing Unit Tests for RITS}
\end{center}

\medskip
\noindent
UTRITS is a domain-specific language that lets you express
\textit{unit tests} for RITS. A unit test is a sequence of RITS
actions and client responses, with some possibilities for
abbreviations. The elements of UTRITS are described in
Table~\ref{tab:utrits}.

\begin{table}[ht]
  \centering
  \begin{tabular}{cp{8.5cm}}
    \hline
    \textit{String} & True when RITS ``emits'' (via RITS action \textbf{format}) a string that contains \textit{String}. \\
    \textbf{solve(Task)} & Start solving \textbf{Task}. This can be initiated either by RITS itself (for example, when spawning a subproblem), or by the client.\\
    \textbf{$=>$(T)} & Respond to the preceding RITS action, which must have been \textbf{read\_answer}, with the client action~\textbf{student\_answers(T)}. \\
    \textbf{*} & The precise meaning of \textbf{*} depends on the next UTRITS element~$E$ in the sequence. In all cases, \textbf{*} means to \textit{ignore} all RITS actions, until:
    \begin{itemize}
    \item If $E$ has the form \textbf{$=>$(T)} and the next RITS action is \textbf{read\_answer}, in which case \textbf{student\_answers(T)} is sent to RITS and normal interaction is resumed.
    \item If $E$ is a literal string and RITS emits a string that contains~$E$, in which case normal interaction is resumed.
    \item If there is no next UTRITS element, then all further interaction is ignored.
    \end{itemize}\\
    \hline
  \end{tabular}
\caption{Admissible elements of UTRITS}
  \label{tab:utrits}
\end{table}

A sample unit test and its result is shown in Fig.~\ref{snip:utrits}.
Notice how messaging flips between server and client, just like in
a real-world example of using RITS.

Due to the side-effect free and declarative nature of LORITS, it is
easy to detect when users must be queried for a response, and to
simulate user input. Unit tests that cover many possible interaction
patterns can therefore be formulated quickly and conveniently with
UTRITS.


\begin{figure}[ht]
  \centering
  \pslisting{utrits.ps}
  \caption{Sample unit test with UTRITS and the resulting interaction}
  \label{snip:utrits}
\end{figure}

\vfil
\end{document}

