\documentclass[a4paper,11pt]{article}
\usepackage[utf8]{inputenc}
\begin{document}
\noindent
\begin{center}
\Large\textbf{RITS System Architecture}
\end{center}

\vspace{1cm}
\noindent
To use RITS, add the following line to your Prolog code:

\bigskip
\textbf{:- use\_module(rits\_engine).}
\bigskip

\noindent RITS is accessible via two API calls:

\begin{itemize}
\item \textbf{rits\_start(-Start)}\\\textbf{Start} is unified with a
  Prolog term that represents the initial state of RITS, before the
  system has received any user actions.

\item \textbf{rits\_next\_action(+A0, -A, +S0, -S)}\\
  This predicate relates a user action \textbf{A0} and a RITS state
  \textbf{S0} to the RITS action \textbf{A} and next state~\textbf{S}.
\end{itemize}

These predicates are completely \textit{pure}: They no not emit any
output, do not write any files, and do not query the user in any way.
Therefore, they can be quite safely executed in a hosting environment.

\vspace{0.7cm}

Due to their purity, these predicates are also very amenable to
regression testing and further analysis. (\textit{more to follow})

\vspace{0.7cm}

The RITS engine may be accessed directly via Prolog, remotely via
JavaScript and \textit{pengines}, or via any other language embedding.

\vfil
\end{document}

